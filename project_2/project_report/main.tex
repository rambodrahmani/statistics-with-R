\documentclass[11pt,a4paper]{article}

\usepackage[a4paper, portrait, margin=1.1in]{geometry}
\usepackage[dvipsnames]{xcolor}
\usepackage[linktoc=none]{hyperref}
\hypersetup{
	colorlinks=true,
	linkcolor=blue,
	filecolor=magenta,      
	urlcolor=blue,
}

\usepackage{listings}
\usepackage{float}
\usepackage{graphicx}
\usepackage[justification=centering]{caption}
\usepackage{wrapfig}
\usepackage{amsmath}

\renewcommand{\contentsname}{Indice}

\definecolor{anti-flashwhite}{rgb}{0.95, 0.95, 0.96}

\begin{document}

\begin{center}
	\Large\textbf{Classificazione dei Supercomputer}\\
	\vspace{0.2cm}
	\large{Progetto per il corso di Statistica del Prof. Marco Romito}\\
	\vspace{0.5cm}
	\large\textit{Rambod Rahmani}\\
	\vspace{0.2cm}
	\scriptsize{Corso di Laurea Magistrale in\\Artificial Intelligence and
	Data Engineering}\\
	\vspace{0.5cm}
	\normalsize{13 Dicembre 2020}
\end{center}

\tableofcontents

\section{Introduzione}
Lo scopo della presente analisi \`e quello di costruire un modello di
classificazione per poter determinare il segmento di mercato di appartenenza di
una Supercomputer a partire dalle specifiche delle sue caratteristiche hardware
e dalle prestazioni ottenute nei principali benchmarks utilizzati in questo
settore.
A partire dalla tabella dei dati, tramite l'utilizzo di R, a seguito di una
preliminare analisi delle componenti principali, sono stati valutati un modello
di classificazione discriminante lineare, uno di classificazione discriminante
quadratica e uno di classificazione con la regressione logistica.\\
\\
Per quanto riguarda il contesto applicativo ipotizzato, possiamo immaginarci che
un tale modello di classificazione possa essere utilizzato, al momento
dell'installazione di un nuovo Supercomputer, per individuare la fascia di
mercato pi\`u idonea in base alle sue prestazioni.

\section{Dati}
La tabella dei dati \`e stata scaricata dal sito dell'organizzazione
\textbf{TOP500}. La TOP500 mantiene una graduatoria, ordinata secondo le loro
prestazioni, dei Supercomputer attualmente installati e in funzione. Tale
graduatoria viene aggiornata con cadenza semestrale.\\
\textbf{Link di download diretto:} \url{https://www.top500.org/lists/top500/2020/11/download/TOP500_202011.xlsx}\\
\begin{figure}[h]
	\vspace{-1cm}
	\begin{minipage}{.3\textwidth}
		\textbf{Credenziali di accesso:}
	\end{minipage}
	\begin{minipage}{0.7\textwidth} 
		\begin{lstlisting}[language=bash,tabsize=2,backgroundcolor=\color{Goldenrod}]
		Login: rambodrahmani@yahoo.it
		Password: GCgFH6yuZYFMeCr
		\end{lstlisting}
	\end{minipage}
	\vspace{-1cm}
\end{figure}
\subsection{Contenuto della tabella}
La tabella dei dati contiene $37$ colonne per un totale di $500$ osservazioni.
Per la presente analisi ho utilizzato le seguenti colonne: \textbf{Site},
\textbf{Manufacturer}, \textbf{Country}, \textbf{Year}, \textbf{Segment},
\textbf{TotalCores}, \textbf{Rmax}, \textbf{Rpeak}, \textbf{Nmax},
\textbf{HPCG}, \textbf{Architecture}, \textbf{Processor},
\textbf{ProcessorTechnology}, \textbf{ProcessorSpeed}, \textbf{OperatingSystem},
\textbf{CoProcessor}, \textbf{CoresPerSocket}, \textbf{ProcessorGeneration},
\textbf{SystemModel}, \textbf{SystemFamily}, \textbf{InterconnectFamily},
\textbf{Interconnect}, \textbf{Continent}. A parte le colonne di significato
ovvio, penso sia doveroso fornire maggiori informazioni i seguenti fattori:
\begin{itemize}
	\setlength\itemsep{0mm}
	\item \textbf{Rmax [TFlop/s]}: massime prestazioni raggiunte nel
		benchmark LINPACK;
	\item \textbf{Rpeak [TFlop/s]}: massime prestazioni teoriche;
	\item \textbf{Nmax}: dimensione del problema sul quale \`e stato
		raggiunto il punteggio Rmax.
	\item \textbf{HPCG [TFlop/s]}: massime prestazioni raggiunte nel
		benchmark HPCG (High Performance Conjugate Gradient);
\end{itemize}
\subsection{Importazione e pulizia}
Sui dati, non \`e stata effettuata alcuna operazione precedente la loro
importazione in R. Il file originale, in formato \texttt{.xlsx}, \`e stato
per\`o convertito in \texttt{.csv} per facilitare l'importazione.\\
Prima di iniziare l'analisi, ho rimosso le colonne che ritengo che non
influenzano la classificazione del segmento di mercato di un Supercomputer
(Rank TOP500, "Name", "Computer", "Power.Source", "OS.Family", ecc\dots), mentre
come fattore per la classificazione \`e stato utilizzato il valore della colonna
"Segment".
\begin{lstlisting}[language=bash,basicstyle=\footnotesize,tabsize=2,frame = single]
> with(data, table(Segment))
Segment
  Academic Government   Industry     Others   Research     Vendor 
        67         34        273         14        103          9
\end{lstlisting}
Nelle colonne che ho scelto, ho rilevato $351$ valori mancanti in
"Accelerator/Co-Processor Cores", $426$ in "HPCG [TFlop/s]" e $488$ in
"Nhalf". Dato che il numero di valori mancanti \`e elevato rispetto al
totale delle $500$ osservazioni, le suddette colonne sono state eliminate.

\section{Analisi}
Subito dopo l'importazione e la pulizia dei dati, sono state effettuate due
preliminari classificazioni ottenendo i seguenti risultati:
\begin{itemize}
	\item \textbf{Analisi Discriminante Lineare}: una accuratezza non
		soddisfacente del $73.33\%$
	\begin{lstlisting}[language=bash,basicstyle=\scriptsize,tabsize=2,frame = single]
Confusion Matrix and Statistics

              Reference
Prediction    Academic   Government   Industry Others Research Vendor
  Academic          42            2          8      0       22      2
  Government         4           16          1      0        2      0
  Industry           5            6        245     11       17      0
  Others             0            0          7      2        1      0
  Research          15           10          8      1       53      1
  Vendor             1            0          4      0        4      5

Overall Statistics
               Accuracy : 0.7333
	\end{lstlisting}
	\item \textbf{Analisi Discriminante Quadratica}: stiamo utilizzando $22$
		fattori con un numero di osservazioni pari a $495$
	\begin{lstlisting}[language=bash,basicstyle=\scriptsize,tabsize=2,frame = single]
Error in qda.default(x, grouping, ...) : 
  some group is too small for 'qda'
	\end{lstlisting}
		Una analisi delle componenti principali per ridurre la
		dimensione del problema si rivela quindi inevitabile.
\end{itemize}
\subsection{Preliminare analisi delle Componenti Principali}
Una prima analisi delle componenti principali, non prendendo in considerazione
la classe delle osservazioni, nonostante ci permetta di ridurre il numero di
fattori presi in considerazione, porta a risultati persino peggiori:
l'accuratezza scende al $68\%$. Ho quindi valutato un secondo modello di PCA
dove ho preso in considerazione anche la classe delle osservazioni.
\subsection{Classificazione per mezzo di Analisi Discriminante Lineare}
\subsection{Classificazione per mezzo di Analisi Discriminante Quadratica}
\subsection{Classificazione per mezzo di Regressione Logistica}
\section{Conclusioni}
In ultima analisi
\end{document}
