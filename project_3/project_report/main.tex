\documentclass[11pt,a4paper]{article}

\usepackage[a4paper, portrait, margin=1.1in]{geometry}
\usepackage[dvipsnames]{xcolor}
\usepackage[linktoc=none]{hyperref}
\hypersetup{
	colorlinks=true,
	linkcolor=blue,
	filecolor=magenta,      
	urlcolor=blue,
}

\usepackage{listings}
\usepackage{float}
\usepackage{graphicx}
\usepackage[justification=centering]{caption}
\usepackage{wrapfig}
\usepackage{amsmath}

\renewcommand{\contentsname}{Indice}

\definecolor{anti-flashwhite}{rgb}{0.95, 0.95, 0.96}

\begin{document}

\begin{center}
	\Large\textbf{Analisi dei commits della repository del Kernel Linux}\\
	\vspace{0.2cm}
	\large{Progetto per il corso di Statistica del Prof. Marco Romito}\\
	\vspace{0.5cm}
	\large\textit{Rambod Rahmani}\\
	\vspace{0.2cm}
	\scriptsize{Corso di Laurea Magistrale in\\Artificial Intelligence and
	Data Engineering}\\
	\vspace{0.5cm}
	\normalsize{23 Dicembre 2020}
\end{center}

\tableofcontents

\section{Introduzione}
La presente analisi si \`e concentrata sul numero di commits della repository
del Kernel Linux suddivisi per mese dall'inizio del progetto sino ad oggi.
\section{Dati}
I dati sono stati ottenuti utilizzando i log della repository
\textbf{torvalds/linux} forniti direttamente da \textbf{github.com}. Come primo
passo \`e stato effettuato un clone pulito della repository:
\begin{lstlisting}[language=bash,basicstyle=\scriptsize,tabsize=2,frame = single]
$ git clone https://github.com/torvalds/linux.git
\end{lstlisting}
Ho poi ottenuti i log dei commits della repository tramite
\begin{lstlisting}[language=bash,basicstyle=\scriptsize,tabsize=2,frame = single]
$ cd linux
$ git log --encoding=latin-1 --pretty="%at|%aN" > git.log
\end{lstlisting}
che ha prodotto il file \texttt{git.log} contenente tutti i commits con il
seguente formato\\

\end{document}
